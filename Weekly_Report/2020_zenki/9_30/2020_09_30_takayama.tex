%\documentstyle[epsf,twocolumn]{jarticle}       %LaTeX2.09仕様
\documentclass[twocolumn]{jarticle}     %pLaTeX2e仕様

%\usepackage[backend=bibtex, style=numeric]{biblatex}
%\addbibresource{sankou.bib}
%%%%%%%%%%%%%%%%%%%%%%%%%%%%%%%%%%%%%%%%%%%%%%%%%%%%%%%%%%%%%%
%%
%%  基本 バージョン
%%
%%%%%%%%%%%%%%%%%%%%%%%%%%%%%%%%%%%%%%%%%%%%%%%%%%%%%%%%%%%%%%%%
\setlength{\topmargin}{-45pt}
%\setlength{\oddsidemargin}{0cm}
\setlength{\oddsidemargin}{-7.5mm}
%\setlength{\evensidemargin}{0cm}
\setlength{\textheight}{24.1cm}
%setlength{\textheight}{25cm}
\setlength{\textwidth}{17.4cm}
%\setlength{\textwidth}{172mm}
\setlength{\columnsep}{11mm}

\setlength{\intextsep}{8pt}
\setlength{\textfloatsep}{8pt}
\setlength{\floatsep}{1pt}

\kanjiskip=.07zw plus.5pt minus.5pt



%【節がかわるごとに(1.1)(1.2) …(2.1)(2.2)と数式番号をつけるとき】
%\makeatletter
%\renewcommand{\theequation}{%
%\thesection.\arabic{equation}} %\@addtoreset{equation}{section}
%\makeatother

%\renewcommand{\arraystretch}{0.95} 行間の設定

\usepackage[dvipdfmx]{graphicx}   %pLaTeX2e仕様(\documentstyle ->\documentclass)
\usepackage{scalefnt}
\usepackage{bm}
\usepackage{here}
\usepackage{url}
\usepackage{amsmath}
\usepackage{amsfonts}
\usepackage[subrefformat=parens]{subcaption}
\captionsetup{compatibility=false}
%%%%%%%%%%%%%%%%%%%%%%%%%%%%%%%%%%%%%%%%%%%%%%%%%%%%%%%%
\usepackage{comment}
\usepackage{subcaption}
\usepackage{multirow}
\usepackage{nidanfloat}
\usepackage[dvipdfmx]{hyperref}

\usepackage[normalem]{ulem}
\useunder{\uline}{\ul}{}

\begin{document}

\twocolumn[
\noindent
\hspace{1em}

令和2年9月30日(水) ゼミ資料
\hfill
\ \ B4 高山 裕成

\vspace{2mm}
\hrule
\begin{center}
{\Large  進捗報告}
\end{center}
\hrule
\vspace{3mm}
]

\section{あらすじ}
hottoSNS-BERT モデルが届いた.

% \footnotesize
\section{進捗}

\begin{itemize}
  \item Pytorch モデルへの変換
  \item 未知語割合測定
\end{itemize}

\section{hottoSNS-BERT}
\begin{verbatim}
  UMR:\DataSet\Model
\end{verbatim}

内に TensorFlow 版と変換した Pytorch 版をそれぞれ置いている.

\section{データセットに含まれる未知語割合}

BERT の事前学習済モデル(京大BERT・hottoSNS-BERT) を用いて 4 コマ漫画ストーリーデータセットについてボキャブラリーに含まれているかどうかを各タッチについてデータオーギュメンテーション前後で算出した.
形態素解析には Juman++ を用いている.

\begin{table*}[htb]
\begin{center}
\caption{データセットに含まれる未知語割合}
\scalebox{0.8}{
\begin{tabular}{lllccccc}
\hline
\multicolumn{1}{c}{} &               &      & \multirow{2}{*}{ギャグ} & \multirow{2}{*}{少女} & \multirow{2}{*}{少年} & \multirow{2}{*}{青年} & \multirow{2}{*}{萌え系} \\
\multicolumn{1}{c}{} &               &      &                      &                     &                     &                     &                      \\ \hline
\multicolumn{2}{l}{拡張前}              &      &                      &                     &                     &                     &                      \\
                     & 京大BERT        & 総単語数 & 311                  & 331                 & 315                 & 316                 & 316                  \\
                     &               & 未知語割合 & 0.026                & 0.021               & 0.019               & 0.022               & 0.025                \\ \cline{2-8}
                     & hottoSNS-BERT & 総単語数 & 270                  & 289                 & 274                 & 276                 & 276                  \\
                     &               & 未知語割合 & 0.133                & 0.131               & 0.128               & 0.127               & 0.120                \\ \hline
拡張後                  &               &      &                      &                     &                     &                     &                      \\
                     & 京大BERT        & 総単語数 & 2633                 & 2728                & 2664                & 2705                & 2708                 \\
                     &               & 未知語割合 & 0.090                & 0.089               & 0.088               & 0.089               & 0.089                \\ \cline{2-8}
                     & hottoSNS-BERT & 総単語数 & 3030                 & 3209                & 3089                & 3154                & 3154                 \\
                     &               & 未知語割合 & 0.695                & 0.700               & 0.696               & 0.698               & 0.697
\end{tabular}
\label{tab:unknown}
}
\end{center}
\end{table*}


\subsection{考察}
\subsubsection{語彙数}
拡張前は京大BERTの方が多かったが拡張後はhottoSNS-BERTの方が多かった.
理由としてはトークナイズ時の仕様の違いによるものだと考えられる.
\begin{itemize}
  \item 京大BERT は 形態素からサブワードへの変換は BPE を用いている
  \item hottoSNS-BERT は形態素解析を行わずSentencePieceを用いて直接テキストからサブワード化している
\end{itemize}
\subsubsection{未知語割合}
前節の理由から例えば形態素「ただいま~」について\\
hottoSNS-BERT では
\begin{verbatim}
ただいま~
\end{verbatim}
のままで全体として未知語として扱われているが京大BERTでは
\begin{verbatim}
ただ ##い ##ま ##〜
\end{verbatim}
と扱われ, その差が未知語割合に大きく差をもたらしていると考えられる.

\subsection{SentencePieceによる形態素解析}
hottoSNS-BERTモデルのみについて, hottoSNS-BERT の分かち書きコーパス構築時に学習されたSentencePieceの事前学習済みモデルを用いて形態素解析した場合についても未知語割合を算出した.

\begin{table*}[htb]
\begin{center}
\caption{SentencePieceによる形態素解析}
\scalebox{0.8}{
\begin{tabular}{lllccccc}
\hline
\multicolumn{2}{c}{\multirow{2}{*}{hottoSNS-BERT}} &      & \multirow{2}{*}{ギャグ} & \multirow{2}{*}{少女} & \multirow{2}{*}{少年} & \multirow{2}{*}{青年} & \multirow{2}{*}{萌え系} \\
\multicolumn{2}{c}{}                               &      &                      &                     &                     &                     &                      \\ \hline
\multicolumn{2}{l}{拡張前}                            &      &                      &                     &                     &                     &                      \\
                 & Juman++ 形態素化                    & 総単語数 & 270                  & 289                 & 274                 & 276                 & 276                  \\
                 &                                 & 未知語割合 & 0.133                & 0.131               & 0.128               & 0.127               & 0.120                \\ \cline{2-8}
                 & SentencePiece 形態素化              & 総単語数 & 318                  & 342                 & 327                 & 332                 & 329                  \\
                 &                                 & 未知語割合 & 0.012                & 0.012               & 0.009               & 0.012               & 0.012                \\ \hline
拡張後              &                                 &      &                      &                     &                     &                     &                      \\
                 & Juman++ 形態素化                    & 総単語数 & 3030                 & 3209                & 3089                & 3154                & 3154                 \\
                 &                                 & 未知語割合 & 0.695                & 0.700               & 0.696               & 0.698               & 0.697                \\ \cline{2-8}
                 & SentencePiece 形態素化              & 総単語数 & 2495                 & 2597                & 2523                & 2568                & 2563                 \\
                 &                                 & 未知語割合 & 0.036                & 0.035               & 0.035               & 0.035               & 0.035
\end{tabular}
\label{tab:unknownspm}
}
\end{center}
\end{table*}

\subsection{考察}
\subsubsection{未知語割合}
Juman++ と比べて大幅に改善され, 拡張前後どちらの場合についても京大BERTよりも未知語割合は下がった.
\subsubsection{重要な未知語}
\begin{verbatim}
['A', 'DVD', 'B', 'GPU']
\end{verbatim}
\subsubsection{データ拡張由来の未知語}
\begin{verbatim}
  ['坐臥', '斯', '迚', '爾', '寔', '悉', '頗', '假', '遁', '顏', '佞', '兇', '譎', '奸譎', '瀟', '窈窕', '婉', '嬋', '悍', '裳', '艷', '閨', '嚊', '嬶', '庸', '灑', '滌', '渫', '臍', '浚', '婢', '嘱', '饒', '噺', '恤', '賦', '婬', '婬靡', '靡', '芍', '辯', '僉', '虞', '尤', '緻', '禀', '稟', '稟賦', '忽', '儕', '儷', '豁', '帙', '韋', '柢', '繙', '誦', '帛', '稍', '聊', '纔', '做', '搦', '疵瑕', '瑕疵', '疵', '昊', '穹', '稗', '躬', '輓', '覯', '扣', '撥', '斥', '洩', '擯斥', '兌', '壟', '渥', '乍', '軈', '孰', '已', '疚']

\end{verbatim}
\section{今後の実験予定}
\begin{itemize}
  \item 前期研究分をhottoSNS-BERTで再実験
  \item 二分割交差検証
  \item データセット不備への対処(ラベルなしセリフをスキップしてデータから除外されていた)
  \item セリフからの話者推定
\end{itemize}

\bibliographystyle{unsrt}
\bibliography{sankou}


\end{document}
