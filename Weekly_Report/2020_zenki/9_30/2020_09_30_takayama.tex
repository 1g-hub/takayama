%\documentstyle[epsf,twocolumn]{jarticle}       %LaTeX2.09仕様
\documentclass[twocolumn]{jarticle}     %pLaTeX2e仕様

%\usepackage[backend=bibtex, style=numeric]{biblatex}
%\addbibresource{sankou.bib}
%%%%%%%%%%%%%%%%%%%%%%%%%%%%%%%%%%%%%%%%%%%%%%%%%%%%%%%%%%%%%%
%%
%%  基本 バージョン
%%
%%%%%%%%%%%%%%%%%%%%%%%%%%%%%%%%%%%%%%%%%%%%%%%%%%%%%%%%%%%%%%%%
\setlength{\topmargin}{-45pt}
%\setlength{\oddsidemargin}{0cm}
\setlength{\oddsidemargin}{-7.5mm}
%\setlength{\evensidemargin}{0cm}
\setlength{\textheight}{24.1cm}
%setlength{\textheight}{25cm}
\setlength{\textwidth}{17.4cm}
%\setlength{\textwidth}{172mm}
\setlength{\columnsep}{11mm}

\setlength{\intextsep}{8pt}
\setlength{\textfloatsep}{8pt}
\setlength{\floatsep}{1pt}

\kanjiskip=.07zw plus.5pt minus.5pt



%【節がかわるごとに(1.1)(1.2) …(2.1)(2.2)と数式番号をつけるとき】
%\makeatletter
%\renewcommand{\theequation}{%
%\thesection.\arabic{equation}} %\@addtoreset{equation}{section}
%\makeatother

%\renewcommand{\arraystretch}{0.95} 行間の設定

\usepackage[dvipdfmx]{graphicx}   %pLaTeX2e仕様(\documentstyle ->\documentclass)
\usepackage{scalefnt}
\usepackage{bm}
\usepackage{here}
\usepackage{url}
\usepackage{amsmath}
\usepackage{amsfonts}
\usepackage[subrefformat=parens]{subcaption}
\captionsetup{compatibility=false}
%%%%%%%%%%%%%%%%%%%%%%%%%%%%%%%%%%%%%%%%%%%%%%%%%%%%%%%%
\usepackage{comment}
\usepackage{subcaption}
\usepackage{multirow}
\usepackage{nidanfloat}
\usepackage[dvipdfmx]{hyperref}

\usepackage[normalem]{ulem}
\useunder{\uline}{\ul}{}

\begin{document}

\twocolumn[
\noindent
\hspace{1em}

令和2年9月30日(水) ゼミ資料
\hfill
\ \ B4 高山 裕成

\vspace{2mm}
\hrule
\begin{center}
{\Large  進捗報告}
\end{center}
\hrule
\vspace{3mm}
]

\section{あらすじ}
hottoSNS-BERT モデルが届いた.

% \footnotesize
\section{進捗}

\begin{itemize}
  \item 未知語率測定
\end{itemize}

\section{データセットに含まれる未知語率}

BERT の事前学習済モデル(京大BERT・hottoSNS-BERT) を用いて 4 コマ漫画ストーリーデータセットについてボキャブラリーに含まれているかどうかを各タッチについてデータオーギュメンテーション前後で算出した.
形態素解析には Juman++ を用いている.

\subsubsection{形態素(Juman++)}
従来, 実験で用いていた手法における未知語率を表したのが表\ref{tab:unknown} である.
拡張前で約 25\%, 拡張後で約 68\% であった.

\begin{table*}[htb]
\begin{center}
\caption{データセットに含まれる未知語率}
\scalebox{0.8}{
\begin{tabular}{lllccccc}
\hline
\multicolumn{1}{c}{} &               &      & \multirow{2}{*}{ギャグ} & \multirow{2}{*}{少女} & \multirow{2}{*}{少年} & \multirow{2}{*}{青年} & \multirow{2}{*}{萌え系} \\
\multicolumn{1}{c}{} &               &      &                      &                     &                     &                     &                      \\ \hline
\multicolumn{2}{l}{拡張前}              &      &                      &                     &                     &                     &                      \\
                     & 京大BERT        & 総単語数 & 311                  & 331                 & 315                 & 316                 & 316                  \\
                     &               & 未知語率 & 0.026                & 0.021               & 0.019               & 0.022               & 0.025                \\ \cline{2-8}
                     & hottoSNS-BERT & 総単語数 & 270                  & 289                 & 274                 & 276                 & 276                  \\
                     &               & 未知語率 & 0.133                & 0.131               & 0.128               & 0.127               & 0.120                \\ \hline
拡張後                  &               &      &                      &                     &                     &                     &                      \\
                     & 京大BERT        & 総単語数 & 2633                 & 2728                & 2664                & 2705                & 2708                 \\
                     &               & 未知語率 & 0.090                & 0.089               & 0.088               & 0.089               & 0.089                \\ \cline{2-8}
                     & hottoSNS-BERT & 総単語数 & 3030                 & 3209                & 3089                & 3154                & 3154                 \\
                     &               & 未知語率 & 0.695                & 0.700               & 0.696               & 0.698               & 0.697
\end{tabular}
\label{tab:unknown}
}
\end{center}
\end{table*}


\subsection{考察}
\subsubsection{語彙数}
拡張前は京大BERTの方が多かったが拡張後はhottoSNS-BERTの方が多かった.
理由としてはトークナイズ時の仕様の違いによるものだと考えられる.
\begin{itemize}
  \item 京大BERT は 形態素からサブワードへの変換は BPE を用いている
  \item hottoSNS-BERT は形態素解析を行わずSentencePieceを用いて直接テキストからサブワード化している
\end{itemize}
\subsubsection{未知語率}
前節の理由から例えば形態素「ただいま~」について\\
hottoSNS-BERT では
\begin{verbatim}
ただいま~
\end{verbatim}
のままで全体として未知語として扱われているが京大BERTでは
\begin{verbatim}
ただ ##い ##ま ##〜
\end{verbatim}
と扱われ, その差が未知語率に大きく差をもたらしていると考えられる.


\section{今後の実験予定}
\begin{itemize}
  \item Juman++ ではなくSentencePieceの事前学習済みモデルを用いて形態素解析を行う.
  \item セリフからの話者推定
\end{itemize}

\bibliographystyle{unsrt}
\bibliography{sankou}


\end{document}
