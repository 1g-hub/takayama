%\documentstyle[epsf,twocolumn]{jarticle}       %LaTeX2.09仕様
\documentclass[twocolumn]{jarticle}     %pLaTeX2e仕様

%\usepackage[backend=bibtex, style=numeric]{biblatex}
%\addbibresource{sankou.bib}
%%%%%%%%%%%%%%%%%%%%%%%%%%%%%%%%%%%%%%%%%%%%%%%%%%%%%%%%%%%%%%
%%
%%  基本 バージョン
%%
%%%%%%%%%%%%%%%%%%%%%%%%%%%%%%%%%%%%%%%%%%%%%%%%%%%%%%%%%%%%%%%%
\setlength{\topmargin}{-45pt}
%\setlength{\oddsidemargin}{0cm}
\setlength{\oddsidemargin}{-7.5mm}
%\setlength{\evensidemargin}{0cm}
\setlength{\textheight}{24.1cm}
%setlength{\textheight}{25cm}
\setlength{\textwidth}{17.4cm}
%\setlength{\textwidth}{172mm}
\setlength{\columnsep}{11mm}

\setlength{\intextsep}{8pt}
\setlength{\textfloatsep}{8pt}
\setlength{\floatsep}{1pt}

\kanjiskip=.07zw plus.5pt minus.5pt



%【節がかわるごとに(1.1)(1.2) …(2.1)(2.2)と数式番号をつけるとき】
%\makeatletter
%\renewcommand{\theequation}{%
%\thesection.\arabic{equation}} %\@addtoreset{equation}{section}
%\makeatother

%\renewcommand{\arraystretch}{0.95} 行間の設定

\usepackage[dvipdfmx]{graphicx}   %pLaTeX2e仕様(\documentstyle ->\documentclass)
\usepackage{scalefnt}
\usepackage{bm}
\usepackage{here}
\usepackage{url}
\usepackage{amsmath}
\usepackage{amsfonts}
\usepackage[subrefformat=parens]{subcaption}
\captionsetup{compatibility=false}
%%%%%%%%%%%%%%%%%%%%%%%%%%%%%%%%%%%%%%%%%%%%%%%%%%%%%%%%
\usepackage{comment}
\usepackage{subcaption}
\usepackage{multirow}
\usepackage{nidanfloat}
\usepackage[dvipdfmx]{hyperref}

\usepackage[normalem]{ulem}
\useunder{\uline}{\ul}{}

\begin{document}

\twocolumn[
\noindent
\hspace{1em}

令和2年5月27日(水) ゼミ資料
\hfill
\ \ B4 高山 裕成

\vspace{2mm}
\hrule
\begin{center}
{\Large  進捗報告}
\end{center}
\hrule
\vspace{3mm}
]

\section{あらすじ}
BERT\cite{BERT} pretrained モデルでの未知語数が分かると Data Augmentation 改善案が浮かぶかもしれない.

% \footnotesize
\section{進捗}

\begin{itemize}
  \item 正例ラベルを変えた時の感情推定
  \item オリジナル ・ 拡張後における未知語数
\end{itemize}

\section{正例ラベルを変えた時の感情推定}

これまでの実験では正例ラベルとして喜楽としていたが, 驚愕・ニュートラルに変えて先週と同様の実験を行った. 実験の結果を表\ref{tab:result} に示す. 特に, 正例ラベルを驚愕とした場合は正例とまったく判断されなかった. これは, 訓練時点でラベル比が極端に正例が少なく, そもそも特徴を学習できていなかった可能性が高い.


\begin{table*}[!b]
\begin{center}
\caption{result}
\scalebox{0.65}{
\begin{tabular}{lllllllllllllllll|lll}
\hline
\multicolumn{2}{c}{\multirow{2}{*}{model}} & \multicolumn{3}{c}{ギャグ} & \multicolumn{3}{c}{少女漫画} & \multicolumn{3}{c}{少年漫画} & \multicolumn{3}{c}{青年漫画} & \multicolumn{3}{c|}{萌え系} & \multicolumn{3}{c}{5タッチ平均} \\
\multicolumn{2}{c}{} & \multicolumn{1}{c}{Acc} & \multicolumn{1}{c}{Recall} & \multicolumn{1}{c}{F1} & \multicolumn{1}{c}{Acc} & \multicolumn{1}{c}{Recall} & \multicolumn{1}{c}{F1} & \multicolumn{1}{c}{Acc} & \multicolumn{1}{c}{Recall} & \multicolumn{1}{c}{F1} & \multicolumn{1}{c}{Acc} & \multicolumn{1}{c}{Recall} & \multicolumn{1}{c}{F1} & \multicolumn{1}{c}{Acc} & \multicolumn{1}{c}{Recall} & \multicolumn{1}{c|}{F1} & \multicolumn{1}{c}{Acc} & \multicolumn{1}{c}{Recall} & \multicolumn{1}{c}{F1} \\ \hline
驚愕 & BERT (Last Layer) & 0.758 & 0.000 & 0.000 & 0.806 & 0.000 & 0.000 & {\ul 0.859} & 0.000 & 0.000 & 0.662 & 0.647 & {\ul 0.500} & {\ul 0.719} & 0.000 & 0.000 & {\ul 0.761} & 0.129 & 0.100 \\
ニュートラル & BERT (Last Layer) & 0.379 & 0.400 & 0.281 & {\ul 0.881} & 0.000 & 0.000 & 0.578 & 0.788 & {\ul 0.658} & 0.677 & 0.200 & 0.276 & 0.625 & 0.267 & 0.250 & 0.628 & 0.331 & 0.293 \\ \hline
喜楽 & BERT (Last Layer) & {\ul 0.833} & 0.400 & {\ul 0.421} & 0.567 & 0.579 & {\ul 0.603} & 0.797 & 0.083 & 0.133 & {\ul 0.800} & 0.357 & 0.435 & 0.656 & 0.455 & {\ul 0.476} & 0.731 & 0.375 & {\ul 0.414} \\ \hline
\multicolumn{2}{l}{ベースライン} & \multicolumn{1}{c}{0.85} & \multicolumn{1}{c}{0} & \multicolumn{1}{c}{0} & \multicolumn{1}{c}{0.43} & \multicolumn{1}{c}{0} & \multicolumn{1}{c}{0} & \multicolumn{1}{c}{0.81} & \multicolumn{1}{c}{0} & \multicolumn{1}{c}{0} & \multicolumn{1}{c}{0.78} & \multicolumn{1}{c}{0} & \multicolumn{1}{c}{0} & \multicolumn{1}{c}{0.66} & \multicolumn{1}{c}{0} & \multicolumn{1}{c|}{0} & \multicolumn{1}{c}{0.71} & \multicolumn{1}{c}{0} & \multicolumn{1}{c}{0}
\end{tabular}
\label{tab:result}
}
\end{center}
\end{table*}

\section{データセットに含まれる未知語率}

BERT の事前学習済モデル\footnote{http://nlp.ist.i.kyoto-u.ac.jp}のボキャブラリーに含まれているかどうかを各タッチについて拡張前後で算出した. その結果が表\ref{tab:unknown} である.

\begin{table}[hb]
\begin{center}
\caption{データセットに含まれる未知語率}
\scalebox{0.8}{
\begin{tabular}{llccccc}
\hline
\multicolumn{2}{c}{\multirow{2}{*}{}} & \multirow{2}{*}{ギャグ} & \multirow{2}{*}{少女} & \multirow{2}{*}{少年} & \multirow{2}{*}{青年} & \multirow{2}{*}{萌え系} \\
\multicolumn{2}{c}{} &  &  &  &  &  \\ \hline
拡張前 & 総単語数 & 270 & 289 & 274 & 276 & 276 \\
 & 未知語率 & 0.215 & 0.239 & 0.208 & 0.210 & 0.203 \\ \hline
拡張後 & 総単語数 & 3030 & 3209 & 3089 & 3154 & 3154 \\
 & 未知語率 & 0.679 & 0.687 & 0.679 & 0.680 & 0.680
\end{tabular}
\label{tab:unknown}
}
\end{center}
\end{table}

\subsection{未知語 (ギャグタッチオリジナル)}
\begin{verbatim}
  'んだろう', 'ございます', '内緒',
  'クールだ', 'ね〜', '顔色',
  '大丈夫です', '昨晩', '手伝える',
  'じゃあ', '添削', 'なあ', 'あー',
  'どうぞ', 'キャー', 'ベタ',
  'なんで', 'おかず',
  'はんぶん', 'こし', 'よっか',
  'ジャーン', 'パフェ', 'ました〜',
  '器用だ', 'いろんな', '飽き',
  '独り占め', '頑張る', 'よそ',
  '人違い', '恥ずかしい',
  'ただいま〜', 'かえり'
\end{verbatim}

\subsection{問題点}
今まで, Juman++ を用いてトークナイズすればいいと思って BERT への入力にそのまま使っていたが,
もしかしたら更に BERT の トークナイザーを用いてサブワードに分けないといけなかったかもしれないので, やり直してきます.
といえども, 未知語率は相当高い.

\section{Data Augmentation 改善案}
\begin{itemize}
  \item 何らかの指標を用いて使用する拡張後のデータを変化させて学習させる.
  \item 正規化辞書(表記ゆれ対策)を作成する.
  \item 意味解析においては Juman は有効だがこのデータセットに対して有効かは疑問. unidic という現代語話し言葉コーパスなども使ってみる.
\end{itemize}

\section{余談}
先週末から, 今使っているノートパソコンの動作がとても不安定になってしまったので昨日, 買い換えました.
なので, USAGI SERV の GPU の警告回りに気を配る余裕がなかったです.

\section{今後の実験予定}
\begin{itemize}
  \item Data Augmentation の手法の改善案の模索.
  \item 直前 n - 1 文 を考慮した n 文を入力して末尾入力の感情推定をする.
\end{itemize}

\bibliographystyle{unsrt}
\bibliography{sankou}


\end{document}
